\documentclass[12pt,]{article}
\usepackage{lmodern}
\usepackage{amssymb,amsmath}
\usepackage{ifxetex,ifluatex}
\usepackage{fixltx2e} % provides \textsubscript
\ifnum 0\ifxetex 1\fi\ifluatex 1\fi=0 % if pdftex
  \usepackage[T1]{fontenc}
  \usepackage[utf8]{inputenc}
\else % if luatex or xelatex
  \ifxetex
    \usepackage{mathspec}
  \else
    \usepackage{fontspec}
  \fi
  \defaultfontfeatures{Ligatures=TeX,Scale=MatchLowercase}
\fi
% use upquote if available, for straight quotes in verbatim environments
\IfFileExists{upquote.sty}{\usepackage{upquote}}{}
% use microtype if available
\IfFileExists{microtype.sty}{%
\usepackage{microtype}
\UseMicrotypeSet[protrusion]{basicmath} % disable protrusion for tt fonts
}{}
\usepackage[margin=1in]{geometry}
\usepackage{hyperref}
\hypersetup{unicode=true,
            pdfauthor={Xuan Son Le (4669361), Freie Universität Berlin},
            pdfborder={0 0 0},
            breaklinks=true}
\urlstyle{same}  % don't use monospace font for urls
\usepackage{color}
\usepackage{fancyvrb}
\newcommand{\VerbBar}{|}
\newcommand{\VERB}{\Verb[commandchars=\\\{\}]}
\DefineVerbatimEnvironment{Highlighting}{Verbatim}{commandchars=\\\{\}}
% Add ',fontsize=\small' for more characters per line
\usepackage{framed}
\definecolor{shadecolor}{RGB}{248,248,248}
\newenvironment{Shaded}{\begin{snugshade}}{\end{snugshade}}
\newcommand{\KeywordTok}[1]{\textcolor[rgb]{0.13,0.29,0.53}{\textbf{#1}}}
\newcommand{\DataTypeTok}[1]{\textcolor[rgb]{0.13,0.29,0.53}{#1}}
\newcommand{\DecValTok}[1]{\textcolor[rgb]{0.00,0.00,0.81}{#1}}
\newcommand{\BaseNTok}[1]{\textcolor[rgb]{0.00,0.00,0.81}{#1}}
\newcommand{\FloatTok}[1]{\textcolor[rgb]{0.00,0.00,0.81}{#1}}
\newcommand{\ConstantTok}[1]{\textcolor[rgb]{0.00,0.00,0.00}{#1}}
\newcommand{\CharTok}[1]{\textcolor[rgb]{0.31,0.60,0.02}{#1}}
\newcommand{\SpecialCharTok}[1]{\textcolor[rgb]{0.00,0.00,0.00}{#1}}
\newcommand{\StringTok}[1]{\textcolor[rgb]{0.31,0.60,0.02}{#1}}
\newcommand{\VerbatimStringTok}[1]{\textcolor[rgb]{0.31,0.60,0.02}{#1}}
\newcommand{\SpecialStringTok}[1]{\textcolor[rgb]{0.31,0.60,0.02}{#1}}
\newcommand{\ImportTok}[1]{#1}
\newcommand{\CommentTok}[1]{\textcolor[rgb]{0.56,0.35,0.01}{\textit{#1}}}
\newcommand{\DocumentationTok}[1]{\textcolor[rgb]{0.56,0.35,0.01}{\textbf{\textit{#1}}}}
\newcommand{\AnnotationTok}[1]{\textcolor[rgb]{0.56,0.35,0.01}{\textbf{\textit{#1}}}}
\newcommand{\CommentVarTok}[1]{\textcolor[rgb]{0.56,0.35,0.01}{\textbf{\textit{#1}}}}
\newcommand{\OtherTok}[1]{\textcolor[rgb]{0.56,0.35,0.01}{#1}}
\newcommand{\FunctionTok}[1]{\textcolor[rgb]{0.00,0.00,0.00}{#1}}
\newcommand{\VariableTok}[1]{\textcolor[rgb]{0.00,0.00,0.00}{#1}}
\newcommand{\ControlFlowTok}[1]{\textcolor[rgb]{0.13,0.29,0.53}{\textbf{#1}}}
\newcommand{\OperatorTok}[1]{\textcolor[rgb]{0.81,0.36,0.00}{\textbf{#1}}}
\newcommand{\BuiltInTok}[1]{#1}
\newcommand{\ExtensionTok}[1]{#1}
\newcommand{\PreprocessorTok}[1]{\textcolor[rgb]{0.56,0.35,0.01}{\textit{#1}}}
\newcommand{\AttributeTok}[1]{\textcolor[rgb]{0.77,0.63,0.00}{#1}}
\newcommand{\RegionMarkerTok}[1]{#1}
\newcommand{\InformationTok}[1]{\textcolor[rgb]{0.56,0.35,0.01}{\textbf{\textit{#1}}}}
\newcommand{\WarningTok}[1]{\textcolor[rgb]{0.56,0.35,0.01}{\textbf{\textit{#1}}}}
\newcommand{\AlertTok}[1]{\textcolor[rgb]{0.94,0.16,0.16}{#1}}
\newcommand{\ErrorTok}[1]{\textcolor[rgb]{0.64,0.00,0.00}{\textbf{#1}}}
\newcommand{\NormalTok}[1]{#1}
\usepackage{graphicx,grffile}
\makeatletter
\def\maxwidth{\ifdim\Gin@nat@width>\linewidth\linewidth\else\Gin@nat@width\fi}
\def\maxheight{\ifdim\Gin@nat@height>\textheight\textheight\else\Gin@nat@height\fi}
\makeatother
% Scale images if necessary, so that they will not overflow the page
% margins by default, and it is still possible to overwrite the defaults
% using explicit options in \includegraphics[width, height, ...]{}
\setkeys{Gin}{width=\maxwidth,height=\maxheight,keepaspectratio}
\IfFileExists{parskip.sty}{%
\usepackage{parskip}
}{% else
\setlength{\parindent}{0pt}
\setlength{\parskip}{6pt plus 2pt minus 1pt}
}
\setlength{\emergencystretch}{3em}  % prevent overfull lines
\providecommand{\tightlist}{%
  \setlength{\itemsep}{0pt}\setlength{\parskip}{0pt}}
\setcounter{secnumdepth}{5}
% Redefines (sub)paragraphs to behave more like sections
\ifx\paragraph\undefined\else
\let\oldparagraph\paragraph
\renewcommand{\paragraph}[1]{\oldparagraph{#1}\mbox{}}
\fi
\ifx\subparagraph\undefined\else
\let\oldsubparagraph\subparagraph
\renewcommand{\subparagraph}[1]{\oldsubparagraph{#1}\mbox{}}
\fi

%%% Use protect on footnotes to avoid problems with footnotes in titles
\let\rmarkdownfootnote\footnote%
\def\footnote{\protect\rmarkdownfootnote}

%%% Change title format to be more compact
\usepackage{titling}

% Create subtitle command for use in maketitle
\newcommand{\subtitle}[1]{
  \posttitle{
    \begin{center}\large#1\end{center}
    }
}

\setlength{\droptitle}{-2em}
  \title{\textbf{Logistische Regression}}
  \pretitle{\vspace{\droptitle}\centering\huge}
  \posttitle{\par}
  \author{Xuan Son Le (4669361), Freie Universität Berlin}
  \preauthor{\centering\large\emph}
  \postauthor{\par}
  \predate{\centering\large\emph}
  \postdate{\par}
  \date{02/04/2018}


\begin{document}
\maketitle

\begin{center}\rule{0.5\linewidth}{\linethickness}\end{center}

\textbf{Abstract}: Im Rahmen der Abschlussarbeit des Moduls
Programmieren mit R im Wintersemester 2017/2018 an der Freie Universität
Berlin wird für diese Arbeit die statistische Methode namens binäres
Logit-Modell ausgewählt. Diese Arbeit besteht aus zwei großen
Hauptteilen: der Theorieteil, wobei die ausgewählte Methode theoretisch
vorgestellt wird und der Implementierungsteil, welcher die Erklärung der
Funktionalität vom selbst entwickelten Paket beinhaltet. Im Theorieteil
wird zunächst ein Überblick über die grundliegende Funktionsweise vom
(binären) Logit-Modell widergegeben. Die Grundidee von Generalisierten
linearen Modellen wird anschließend kurz eingeführt, bevor der Aufbau
vom binären Logit-Modell durch das Maximum Likelihood Verfahren
vorgenommen wird. Demzufolge folgt die Interpretation der Koeffizienten
und der Lösungsgüte vom binären Logit-Modell. Schließlich werden im
Implementierungsteil alle Funktionen vom R-Paket schritterweise
vorgestellt.

\textbf{Keywords:} \emph{Logit-Modell, logistische Regression, Paket, R}

\begin{center}\rule{0.5\linewidth}{\linethickness}\end{center}

\newpage

\section{Motivation}\label{motivation}

Die Anwendung von der klassischen linearen Regression ist für binäre
(binomiale oder dichotome) Zielvariable, welche lediglich zwei Werte
(ja/nein, mänlich/weiblich, erfolgreich/nicht erfolgreich, etc.)
annehmen kann, nicht mehr geeignet, da die Zielvariable von der linearen
Regression metrisch skaliert ist. Oft wird binäre Variable als
0/1-Variable kodiert, das heißt sie nimmt nur den Wert 0 oder 1 an. Die
folgende Grafik stellt den Ansatz graphisch dar, binäre Variable durch
lineare Regression zu modellieren:

\includegraphics{logisticRegression_files/figure-latex/unnamed-chunk-1-1.pdf}

Graphisch lässt sich festlegen, dass die lineare Regression den
Wertebereich {[}0,1{]} von binären Responsevariablen sehr schnell
verlässt. Wenn die Annahmen von der linearen Regression noch in Betracht
gezogen werden, ergeben sich darüber hinaus noch folgende Probleme (vgl.
\ldots{}): \emph{ } *

Aus diesen Gründen wird ein ganz anderer Ansatz benötigt, um binäre
Zielvariable zu modellieren, nämlich das binäre Logit-Modell, welches
ebenfalls als binäre logistische Regression oder binäres logistisches
Regressionsmodell bezeichnet werden kann. In der Statistik lassen sich
Logit-Modelle noch in multinomiale und kumulative Logit-Modelle
aufteilen, je nachdem ob die abhängige Variable multinominal- oder
ordinalskaliert sind . Diese Arbeit beschäftigt sich mit dem binären
Logit-Modell, welches den Zusammenhang zwischen einer binären abhängigen
Variable und einer/mehreren unabhängigen Variablen untersucht. Bei allen
Arten von Logit-Modellen können die unabhängigen Variablen beliebig
skaliert sein.

Im Unterschied zu der klassischen linearen Regression, welche den wahren
Wert einer Zielvariable vorhersagt, interessiert sich das binäre
Logit-Modell eher für die Wahrscheinlichkeit, dass die Zielvariable den
Wert 1 annimmt. Das Hauptziel vom binären Logit-Modell ist es, die
Wahrscheinlichkeit für den Eintritt der Zielvariable vorherzusagen.
Dadurch soll die folgende theoretische Fragestellung beantwortet werden:
\emph{Wie stark ist der Einfluss von den unabhängigen (erklärenden)
Variablen auf die Wahrscheinlichkeit, dass die abhängige (zu erklärende
/ Response) Variable eintritt beziehungsweise den Wert 1 annimmt?} In
der Praxis kann diese Fragestellung beispielsweise so formuliert werden:
``Haben Alter, Geschlecht, Berufe oder andere Merkmale der Kunden
Einfluss auf die Wahrscheinlichkeit, dass sie ein Kredit rechtzeitig
zurückzahlen?'' oder ``Lässt sich die Wahrscheinlichkeit, dass es
regnet, durch die Temparatur, die Windstärke oder
Sonnenstrahlungsintensität vorhersagen?''.

\section{Das binäre Logit-Modell}\label{das-binare-logit-modell}

\subsection{Modellspezifikation}\label{modellspezifikation}

Das Logit-Modell ist eine Methode aus der Algorithmenklasse namens
\emph{Generalisierte Lineare Modelle} (engl. generalized linear model,
kurz GLM), welche eine Verallgemeinerung des klassischen linearen
Regressionsmodells anstrebt. Dazu gehören noch die klassische lineare
Regression, Probitmodell und Poisson-Regression. Die Grundidee von GLM
ist die Transformation der linearen Regressionsgleichung, so dass der
Wertebereich der vorhergesagten Zielvariable dem gewünschten entspricht.
Ein GLM besteht aus drei Hauptelementen: die systematische Komponente,
die Link-Funktion und die Zufallskomponente.

Gegeben seien n unabhängige Beobachtungen \(y_1, y_2, ...,y_n\) der
binären Zielvariable \(\mathbf{Y}\). Ein Verteilungsmodell für
\(\mathbf{Y}\) ist die Binomialverteilung:
\(\mathbf{Y}_i \sim B(1, \pi_i)\) mit \(\pi_i = P(Y_i = 1)\)

Für diese Arbeit wird \(\pi_i = (\pi_1, \pi_2, ..., \pi_n)\) als die
Eintrittwahrscheinlichkeit von der einzelnen \(\mathbf{Y}_i\) benannt.
Weiterhin seien p erklärende Variablen
\(\mathbf{X}_0,\mathbf{X}_1,..,\mathbf{X}_k\) gegeben mit jeweils n
unabhängigen Beobachtungen
\(\mathbf{X}_j = (x_{1j}, x_{2j},..., x_{nj})\) mit j \(\in\)
\{1,2,..,k\} gegeben. Daraus ergeben sich p Koeffizienten
\(\beta = (\beta_0, \beta_1, \beta_2,..., \beta_p)\), welche die Stärke
den Zusammenhang zwischen die einzelne erklärende Variable mit der
Zielvariable widerspiegeln. Dabei ist es sinnvoll, diese in einer
Designmatrix \(\mathbf{X}\) zu speichern. Da der Interzept (\(\beta_0\))
ebenfalls geschätzt werden soll, sind alle Werte der ersten Spalte von X
gleich Eins, also \(x_{10} = x_{20} = ... = x_{n0} = 1\).
Zusammengefasst lässt sich die Designmatrix wie folgt darstellen: \[
\mathbf{X} =
 \begin{pmatrix}
    1 & x_{11} & x_{12} & \cdots & x_{1k} \\
    1 & x_{21} & x_{22} & \cdots & x_{2k} \\
    1 & x_{31} & x_{32} & \cdots & x_{3k} \\
    \vdots  & \vdots  & \vdots & \ddots & \vdots \\
    1 & x_{n1} & x_{n2} & \cdots & x_{nk}
 \end{pmatrix}
\] Die dazugehörige lineare Regressionsgleichung lautet:
\(\mathbf{Y} = \mathbf{X}.\beta + \epsilon\) mit
\(\epsilon = (\epsilon_1, \epsilon_2, \epsilon_3, ..., \epsilon_n)\) als
die Abweichung der einzelnen Schätzungen gegenüber dem wahren Wert. Die
einzelne Beobachtung lässt sich wie folgt darstellen:
\[y_i = \beta_0 + \beta_1.x_{i2} + \beta_2.x_{i3} + ... + \beta_k.x_{ik} + \epsilon_i \qquad \forall_i = 1,2,3,...,n\]

Um die Werte im Bereich der reellen Zahlen von der linearen Regression
auf dem Wertebereich von Wahrscheinlichkeiten zwischen 0 und 1 zu
beschränken, sollte die rechte Seite der Gleichung transformiert werden.
Das Ziel ist es, eine sinnvolle Verteilungsfunktion (Responsefunktion)
zu finden, deren Wertebereich in {[}0,1{]} liegt:
\(\pi_i = F(\beta_0 + \beta_1.x_{i2} + \beta_2.x_{i3} + ... + \beta_k.x_{ik} + \epsilon_i) = F(\eta_i)\).
Der lineare Prädikator
\(\eta_i = \beta_0 + \beta_1.x_{i2} + \beta_2.x_{i3} + ... + \beta_k.x_{ik} + \epsilon_i\)
wird ebenfalls als Linkfunktion genannt, weil dadurch eine Verbindung
(Link) zwischen der Eintrittwahrscheinlichkeit und den unabhängigen
Variablen erfolgt wird.

\subsection{Maximum Likelihood
Schätzung}\label{maximum-likelihood-schatzung}

Während bei der klassischen linearen Regression die Methode der
Kleinsten Quadrate (engl. \emph{method of least squares}) genutzt wird,
um eine Regressionslinie zu bestimmen, welche die Summe der
quadratischen Abweichungen minimiert, wird bei dem binären Logit-Modell
die sogenannte Maximum Likelihood Schätzung eingesetzt.

\subsection{Intepretation der
Koeffizienten}\label{intepretation-der-koeffizienten}

\section{Implementierung in R}\label{implementierung-in-r}

Im Folgenden wird die Funktionalität von dem Paket \textbf{logitModell}
erklärt, welches zum Ziel setzt, die Grundidee hinter dem binären
Logit-Modell programmiert darzustellen. Das Paket besteht aus dem
R-Code, welcher anhand dem manuell berechneten Maximum Likelihood ein
Objekt von der Klasse \emph{logitMod} erstellt und anschließend drei S3
Methoden für diese Klasse (print, summary und plot) definiert, und einer
Vignette, welche den R-Code anhand einem konkreten Beispiel ausführt.
Dieser Beispieldatensatz wird im Folgenden verwendet, um die Richtigkeit
und Vollständigkeit der Ergebnisse der implementierten Methode im
Vergleich zu der R-Standardmethode für Logit-Modell \emph{glm(\ldots{},
family = ``binomial'')} zu testen. Die binäre Responsevariable heißt
\emph{admit}, welche besagt ob ein Kandidat eine Zulassung bekommt.
Zudem enthält der Datensatz drei unabhängige Variablen: \emph{gre},
\emph{gpa} (metrisch) und \emph{rank} (kategorial). Der Datensatz soll
ein Modell unterstützen, welche die Abhängigkeit von der
Wahrscheinlichkeit einer Zulassung von der Abschlussnote, GRE-Note sowie
dem Ruf von der angestrebten Institution.

\subsection{Beispieldatensatz}\label{beispieldatensatz}

Zunächst wird der Datensatz importiert. Dabei wird die Zielvariable aus
dem Datensatz entnommen und in einem Vektor gespeichert. Da diese schon
als 0/1-Variable vorgegeben wird, besteht es in diesem Fall keine
Notwendigkeit, die Zielvariable zu faktorisieren. Der Code funktioniert
allerdings ebenfalls mit Zielvariable, welche zum Beispiel als
weiblich/männlich oder Erfolg/kein Erfolg kodiert wird und transformiert
diese in eine 0/1-Variable.

\begin{Shaded}
\begin{Highlighting}[]
\CommentTok{# sei y die eingegebene Zielvariable}
\ControlFlowTok{if}\NormalTok{ (}\OperatorTok{!}\NormalTok{(}\DecValTok{0} \OperatorTok\StringTok{ }\NormalTok{y }\OperatorTok{&&}\StringTok{ }\DecValTok{1} \OperatorTok\StringTok{ }\NormalTok{y)) \{}
\NormalTok{    y <-}\StringTok{ }\KeywordTok{factor}\NormalTok{(y, }\DataTypeTok{labels =} \KeywordTok{c}\NormalTok{(}\DecValTok{0}\NormalTok{,}\DecValTok{1}\NormalTok{))}
\NormalTok{\}}
\NormalTok{y <-}\StringTok{ }\KeywordTok{as.numeric}\NormalTok{(}\KeywordTok{as.character}\NormalTok{(y))}
\end{Highlighting}
\end{Shaded}

Es muss immer vorab überprüft werden, in welcher Art die Zielvariable
eingegeben wird, denn das Maximum Likelihood braucht als Input
numerische Vektoren für weitere Berechnungen. Dieser Schritt wird extra
gemacht, damit sich das manuelle Modell im Hinblick auf den Input gleich
verhält wie das Standardmodell.

\subsection{Maximum Likelihood
Schätzung}\label{maximum-likelihood-schatzung-1}

Bevor das eigentliche Logit-Modell erstellt wird, wird in diesem
Abschnitt die Implementierung der Maximum Likelihood Schätzung
auseinandergesetzt. Der Code dazu ist auf Basis von dem betroffenen
theoretischen Teil (siehe Abschnitt \ldots{}) aufgebaut. Schrittweise
werden die einzelnen Parameter definiert. Daraus wird in der
Newton-Raphson-Schleife das Maximum Likelihood berechnet.

Das gerade ausgeführte Beispiel kann direkt in R geladen werden. Dafür
wird in das Paket ein Vignette eingebaut, so dass wenn den folgenden
Code ausgeführt wird, wird das Beispiel in der Help-Seite von R
angezeigt.

\begin{Shaded}
\begin{Highlighting}[]
\KeywordTok{setwd}\NormalTok{(}\StringTok{"~/Desktop/Uni/Master/WS1718/ProgR/Abschlussarbeit/logisticRegression/Code/logitModell"}\NormalTok{)}
\NormalTok{devtools}\OperatorTok{::}\KeywordTok{install}\NormalTok{(}\DataTypeTok{build_vignettes =} \OtherTok{TRUE}\NormalTok{)}
\KeywordTok{vignette}\NormalTok{(}\StringTok{"logitModell"}\NormalTok{)}
\end{Highlighting}
\end{Shaded}

\[
\begin{pmatrix}
    y_1 \\ y_2 \\ y_3 \\ \vdots \\ y_n 
 \end{pmatrix} = 
 \begin{pmatrix}
    1 & x_{11} & x_{12} & \cdots & x_{1p} \\
    1 & x_{21} & x_{22} & \cdots & x_{2p} \\
    1 & x_{31} & x_{32} & \cdots & x_{3p} \\
    \vdots  & \vdots  & \vdots & \ddots & \vdots \\
    1 & x_{n1} & x_{n2} & \cdots & x_{np}
 \end{pmatrix} \cdot
 \begin{pmatrix}
    \beta_1 \\ \beta_2 \\ \beta_3 \\ \vdots \\ \beta_k  
 \end{pmatrix} +
 \begin{pmatrix}
    \epsilon_1 \\ \epsilon_2 \\ \epsilon_3 \\ \vdots \\ \epsilon_n  
 \end{pmatrix}
\]


\end{document}
